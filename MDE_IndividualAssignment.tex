\documentclass[a4]{scrartcl}

\usepackage[ngerman]{babel}
\usepackage[utf8]{inputenc}
\usepackage{mathtools}
\usepackage{amsmath}
\usepackage{amssymb}
\usepackage{geometry}
\usepackage{scrlayer-scrpage}
\pagestyle{scrheadings}
\clearscrheadfoot


\geometry{
	paper=a4paper, % Change to letterpaper for US letter
	top=2cm, % Top margin
	bottom=1.5cm, % Bottom margin
	left=2cm, % Left margin
	right=3cm, % Right margin
	%showframe, % Uncomment to show how the type block is set on the page
}

\usepackage[backend=biber, maxbibnames=99]{biblatex}
\addbibresource{references.bib}

%\ohead{\\
%	Pina Kolling\\
%	piko0011}

\begin{document}
	
	\begin{titlepage}
		\centering
		{\scshape\LARGE Umeå University \par}
		\vspace{1cm}
		{\scshape\Large Managing the Digital Enterprise \par }
		\vspace{1.5cm}
		{\huge\bfseries  Individual Assignment 1 \par}
		\vspace{2cm}
		{\Large\itshape Pina Kolling\par}
		\vfill
		supervised by \par 
		\vspace{1cm}
		Daniel \textsc{Skog} \par 
		and \par 
		Ramy \textsc{Shenouda} 
		
		\vfill
		
		% Bottom of the page
		{\large \today\par}
	\end{titlepage}
	
	\setcounter{page}{1}
	
	\tableofcontents
	
	\newpage
	
	% Assignment 1 (For the grade Pass)
	%As digitalization increasingly permeates all levels of society, the conditions, tools and challenges for contemporary organizing are changing. The course literature provides different (but interconnected) perspectives, explanations and analyses of the changing nature of organizing. Your task is to: 
	
	%Use the concepts of scale, scope, and speed to explain how business and organizing conditions are changing for companies as their environments are increasingly digitalized.
	%Explain the concept of "fast movers" and use the concept of the nexus of scale, scope and speed to explain how organizations can become fast movers.
	%Argue for what leadership capabilities and digital capabilities an organization should prioritize building in order to become a fast mover. 
	
	
	%In order to pass, the student needs to: 
	%Present the concepts of scale, scope, speed and their dynamics in a way that corresponds with that of the course literature.  
	%Use the concepts of scale, scope, and speed to clearly demonstrate how conditions for business and organizing change in digitalized envrionments. 
	%Present the concept of fast mover in a way that corresponds with that of the course literature. 
	%Illustrate how the concept of fast movers logically follows from the conditions offered by the nexus of scale, scope, and speed. 
	%Present clear and grounded arguments to which digital and leadership capabilities an organization should focus on and why when it wants to become a fast mover.
	%Instructions:
	%Submit your answer as a .pdf file
	%The answer to the assignment has a limit of 2500 words
	%Read the questions carefully and make sure that your answers meet the stated grading criteria.
	%Be ambitious in terms of language, indication of sources, and structure. The university library has useful tips to assist you with this at https://writingguide.se/
	%Please note that your arguments must be grounded in the course literature (unless otherwise is specified by the question). You are, however, encouraged to make use of additional sources.
	%This is an individual exam and any attempts to plagiarize will be identified by a plagiarism software. Plagiarism in any form will be reported and followed by disciplinary action.
	%For this assignment, students can obtain the grade Pass or Fail.
	%Your results will be communicated no later than 15 working days after assignment deadline.
	%If a student does not obtain Pass after the first attempt, a re-examn will be offered, however not earlier than 10 working days after the grades from the first round have been communicated.
	
	\section{Effects of digitalization}
	
	%Use the concepts of scale, scope, and speed to explain how business and organizing conditions are changing for companies as their environments are increasingly digitalized.
	
	
	\section{Fast movers}
	
	%Explain the concept of "fast movers" and use the concept of the nexus of scale, scope and speed to explain how organizations can become fast movers.
	
	
	
	\section{Requirements to become a fast mover}
	
	%Argue for what leadership capabilities and digital capabilities an organization should prioritize building in order to become a fast mover. 
	
	
	
	
	
	
	\printbibliography
	
	
	
	
	
	
\end{document}