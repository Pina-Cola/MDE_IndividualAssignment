\documentclass[a4]{scrartcl}

% \usepackage[ngerman]{babel}
\usepackage[utf8]{inputenc}
\usepackage{mathtools}
\usepackage{amsmath}
\usepackage{amssymb}
\usepackage{geometry}
\usepackage{scrlayer-scrpage}
\pagestyle{scrheadings}
\clearscrheadfoot

\setlength{\parindent}{0em}

\setlength{\parskip}{1.3ex}

\usepackage[onehalfspacing]{setspace}


\geometry{
	paper=a4paper, % Change to letterpaper for US letter
	top=2cm, % Top margin
	bottom=1.5cm, % Bottom margin
	left=2cm, % Left margin
	right=3cm, % Right margin
	%showframe, % Uncomment to show how the type block is set on the page
}

\usepackage[backend=biber, maxbibnames=99]{biblatex}
\addbibresource{references.bib}





\usepackage[framemethod=TikZ]{mdframed}

% Style %
\mdfdefinestyle{enviStyle}{
	innertopmargin = 10pt,
	linewidth      = 1pt,
	frametitlerule = true,
	roundcorner    = 2pt%
}





\newenvironment{CountingDefinition}[2][]{%
	\ifstrempty{#1}%
	{\mdfsetup{%
			frametitle={{\strut ~}}}
	}%
	{\mdfsetup{%
			frametitle={{\strut ~#1}}}%
	}%
	\mdfsetup{
		nobreak                   = true,
		linecolor                 = gray,
		frametitlebackgroundcolor = gray!50,
		style                     = enviStyle
	}
	\begin{mdframed}[]\relax%
		\label{#2}}{\end{mdframed}}









%\ohead{\\
%	Pina Kolling\\
%	piko0011}

\begin{document}
	
	\begin{titlepage}
		\centering
		{\scshape\LARGE Umeå University \par}
		\vspace{1cm}
		{\scshape\Large Managing the Digital Enterprise \par }
		\vspace{1.5cm}
		{\huge\bfseries  Individual Assignment 1 \par}
		\vspace{2cm}
		{\Large\itshape Pina Kolling\par}
		\vfill
		supervised by \par 
		\vspace{1cm}
		Dr. Daniel \textsc{Skog} \par 
		and \par 
		M. Sc. Ramy \textsc{Shenouda} 
		
		\vfill
		
		% Bottom of the page
		{\large \today\par}
	\end{titlepage}
	
	\setcounter{page}{1}
	
	\tableofcontents
	
	\newpage
	
	% Assignment 1 (For the grade Pass)
	%As digitalization increasingly permeates all levels of society, the conditions, tools and challenges for contemporary organizing are changing. The course literature provides different (but interconnected) perspectives, explanations and analyses of the changing nature of organizing. Your task is to: 
	
	%Use the concepts of scale, scope, and speed to explain how business and organizing conditions are changing for companies as their environments are increasingly digitalized.
	%Explain the concept of "fast movers" and use the concept of the nexus of scale, scope and speed to explain how organizations can become fast movers.
	%Argue for what leadership capabilities and digital capabilities an organization should prioritize building in order to become a fast mover. 
	
	
	%In order to pass, the student needs to: 
	%Present the concepts of scale, scope, speed and their dynamics in a way that corresponds with that of the course literature.  
	%Use the concepts of scale, scope, and speed to clearly demonstrate how conditions for business and organizing change in digitalized envrionments. 
	%Present the concept of fast mover in a way that corresponds with that of the course literature. 
	%Illustrate how the concept of fast movers logically follows from the conditions offered by the nexus of scale, scope, and speed. 
	%Present clear and grounded arguments to which digital and leadership capabilities an organization should focus on and why when it wants to become a fast mover.
	%Instructions:
	%Submit your answer as a .pdf file
	%The answer to the assignment has a limit of 2500 words
	%Read the questions carefully and make sure that your answers meet the stated grading criteria.
	%Be ambitious in terms of language, indication of sources, and structure. The university library has useful tips to assist you with this at https://writingguide.se/
	%Please note that your arguments must be grounded in the course literature (unless otherwise is specified by the question). You are, however, encouraged to make use of additional sources.
	%This is an individual exam and any attempts to plagiarize will be identified by a plagiarism software. Plagiarism in any form will be reported and followed by disciplinary action.
	%For this assignment, students can obtain the grade Pass or Fail.
	%Your results will be communicated no later than 15 working days after assignment deadline.
	%If a student does not obtain Pass after the first attempt, a re-examn will be offered, however not earlier than 10 working days after the grades from the first round have been communicated.












%-------------------------------------------------------------------------------------------
	\section{Effects of digitalization} \label{sec:EffectsOfDigitalization}
	
	%Use the concepts of scale, scope, and speed to explain how business and organizing conditions are changing for companies as their environments are increasingly digitalized.
	
	The digitalization of business and organization has a big impact on companies, reshaping the way they operate and compete. These changes can be analyzed through the lens of \textit{Scale}, \textit{Scope}, and \textit{Speed}. 
	
	\begin{CountingDefinition}[Definitions]{def:defdef}
		
		\begin{tabular}{lp{13.5cm}}
			
			\textbf{Scale} & describes the rate at which products are created.~\cite{scalescopespeedslides, digitalmatrix} \\
			\textbf{Scope} & describes the investment in customer relationships or physical distribution.~\cite{scalescopespeedslides, digitalmatrix} \\
			\textbf{Speed} & describes the pace in which existing capabilities are improved and new capabilities are developed.~\cite{scalescopespeedslides, digitalmatrix} \\
			
		\end{tabular}
	
				
	\end{CountingDefinition}


	\textit{Scale}, \textit{Scope} and \textit{Speed} are also referred to as the three dimensions of digital business.~\cite{scalescopespeedslides, digitalmatrix}









%----------------------------------------------------------------------------------------------
\subsection{Scale} \label{subsec:Scale}
 
\vspace*{1.3em}
\begin{CountingDefinition}[Scale]{def:scale}
	Scale refers to the size and reach of an organization, for example the rate at which products are created. \cite{scalescopespeedslides, digitalmatrix}
\end{CountingDefinition}


Companies extended their \textit{Scale} in the industrial era by increasing sales of products. This was limited by physical attributes such as cost and availability of materials and reaching enough costumers.  \cite{digitalgoods, digitalmatrix, wiwi}


A big aspect of the change of a company's \textit{Scope} in the context of digitalization is the existence of digital goods. Goods can be seperated into the categories of material goods, services and digital goods. Digital goods can easily and cheap be reproduced and distributed. A physical product (material good) has to be produced more often, which generates more costs for material, transport, logistics and production in general. A digital product can often be reproduced and distributed on a bigger scale easily, after the product was created once. Only maintenance aspects like potential server costs can increase -- the initial development costs of the product will not change. \cite{digitalgoods, wiwi}


Digitalization enables companies to reach a global market without a material good.
But digital goods introduce new challenges like a high rate of piracy, because the digital goods can easily be reproduced and distributed by the consumer, too. \cite{digitalgoods}







%----------------------------------------------------------------------------------------------
\subsection{Scope} \label{subsec:Scope}

\begin{CountingDefinition}[Scope]{def:scope}
	Scope describes the investment in customer relationships or	physical distribution. \cite{scalescopespeedslides, digitalmatrix}
\end{CountingDefinition}

As described in Section \ref{subsec:Scale}, companies extended their \textit{Scale} in the industrial era by increasing sales of products. This was limited by physical attributes such as reaching enough costumers or physical distrubution (\textit{Scope}). Digitalization enables companies to reach a global market more easily, because the costs of communication and transportation and distribution were decreased drastically. \cite{digitalgoods, digitalmatrix, wiwi}

Digitalization introduced new opportunities to distribute products - for example the distribution of digital goods. Partnerships between businesses became more important and the relationships to the customer are primarily build online. A website for distribution and online customer support often replace the personal experience of direct communication between a seller and a customer. \cite{digitalgoods, digitalmatrix, wiwi}







%----------------------------------------------------------------------------------------------
\subsection{Speed} \label{subsec:Speed}

\begin{CountingDefinition}[Speed]{def:speed}
	Speed describes the pace in which existing capabilities are improved and new capabilities are developed. \cite{scalescopespeedslides, digitalmatrix}
\end{CountingDefinition}

Digitalization has accelerated the pace of business operations. As already mentioned in Section \ref{subsec:Scale} and \ref{subsec:Scope}, the product development cykles have shortened and become more easy and the existence of digital goods changed the whole market. \cite{ digitalgoods, digitalmatrix}

But with the increase of \textit{Speed} the customer expectations adapted and the expectations for the \textit{Speed} of a company have increased. \cite{wiwi}

	
Overall, digitalization has transformed the business landscape by increasing the potential and rate for \textit{Scale}, expanding the \textit{Scope} of activities, and accelerating the \textit{Speed}. Companies are trying to adapt to these changes by continuously monitoring and adjusting their strategies to suceed in the digital era. Failure to do so can leave them at a disadvantage, because the customer expectations are constantly growing. \cite{digitalmatrix, leadingdigital, wiwi}
	
	
	
	
	
	
	
	
%-------------------------------------------------------------------------------------------	
	
	\section{Fast-movers} \label{sec:FastMovers}
	
	%Explain the concept of "fast movers" and use the concept of the nexus of scale, scope and speed to explain how organizations can become fast movers.
	
	Fast movers have the ability to recognize opportunities and respond to challenges rapidly.~\cite{fastmovers, digitalmatrix}

	\begin{CountingDefinition}[Fast-mover]{def:fastmovers}
		Fast-movers refers to companies or organizations that can adapt quickly to changes in their environment, for example market shifts, technological developments or competitive pressures. \cite{fastmovers, digitalmatrix}
	\end{CountingDefinition}

	The concepts of \textit{Scale}, \textit{Scope}, and \textit{Speed} can be used to explain how organizations become fast-movers. Those concepts were explained in Section \ref{sec:EffectsOfDigitalization}.
	
	\begin{itemize}
		\item \textit{\textbf{Scale:}}
		\item \textit{\textbf{Scope:}}
		\item \textit{\textbf{Speed:}}
	\end{itemize}


	% chat GPT
	
	%The concept of the "nexus of scale, scope, and speed" is a strategic framework that explains how organizations can become fast movers. Let's break down each element of this nexus:
	
	%1. **Scale**: Scale refers to the size and reach of an organization. It encompasses factors like the company's market share, production capacity, and customer base. Fast movers leverage scale to their advantage by having a significant presence in their industry. This allows them to access more resources, both in terms of capital and talent, which can be used to fund and execute rapid initiatives.
	
	%*Example*: Amazon is a classic example of leveraging scale. Its vast distribution network and customer base enable the company to launch new services and products quickly, capitalizing on its existing infrastructure.
	
	%2. **Scope**: Scope refers to the range of activities and markets in which an organization is involved. Fast movers often have a broad scope, allowing them to diversify their revenue streams and reduce risk. A wide scope can also mean access to a variety of markets and customer segments, which can be advantageous in quickly adapting to changing trends.
	
	%*Example*: Alphabet Inc. (Google's parent company) has a wide scope, with interests in search, advertising, cloud computing, and autonomous vehicles. This diversification allows it to adapt to shifts in technology and market demands swiftly.
	
	%3. **Speed**: Speed is perhaps the most crucial element in becoming a fast mover. It involves the ability to make decisions quickly, execute strategies rapidly, and respond to changes in the business environment with agility. Speed often requires a culture of innovation, streamlined decision-making processes, and efficient execution capabilities.
	
	%*Example*: Tesla is known for its speed in developing and rolling out new electric vehicle models and technology updates. The company's focus on rapid innovation has allowed it to stay ahead of competitors in the electric vehicle market.
	
	%Now, how do these elements interact in the nexus of scale, scope, and speed to help organizations become fast movers?
	
	%- **Leveraging Scale and Scope**: Organizations with significant scale and scope can use their resources and market presence to experiment with new ideas, expand into adjacent markets, and diversify their offerings. This diversification provides a buffer against market volatility, and it allows them to take calculated risks.
	
	%- **Speed through Innovation**: To become a fast mover, organizations need to prioritize innovation. They should encourage a culture of creativity, invest in R&D, and develop the ability to quickly prototype and test new ideas. Speed is not just about acting fast but also about acting smartly.
	
	%- **Adaptive Leadership and Decision-Making**: Fast-moving organizations have leaders who are open to change and can make rapid decisions. They often decentralize decision-making to empower teams closer to the front lines, enabling faster responses to market dynamics.
	
	%- **Efficient Execution**: Finally, fast movers excel in executing their strategies efficiently. They have agile processes, supply chain efficiencies, and technology infrastructures that allow them to turn ideas into reality quickly.
	
	%In conclusion, becoming a fast mover requires a strategic balance between scale, scope, and speed. Organizations that can effectively harness these elements can adapt to changing circumstances, seize opportunities, and outmaneuver competitors in today's dynamic business landscape.
	
	
	
	% buch
	%However, scale at speed creates not first-mover advantage but
	%fast-mover advantage, which may currently be limited by your company’s
	%internal organizational processes and systems, if they cannot recognize and
	%respond to the shifts as quickly as some of the newer companies. Changing
	%scope at speed also reflects fast-mover advantage, where the advantage may
	%lie not necessarily in launching products but in tapping into scarce critical
	%resources such as unique interconnected data, patents, talents, or research
	%and development projects, often executed with others.
	%How well you stack up against not only other incumbents, who themselves
	%are transforming, but also against newer-age companies that are aiming to
	%disrupt and transform your industry may well define your ability to compete and
	%win in the digital realm. Those companies that take maximum advantage of
	%scale, scope, and speed together are able to gain significant advantage in the
	%digital business world. First, with data and analytics and connectivity, you can
	%now extend your footprint beyond your core firm’s boundaries and tap into
	%extended ecosystems. Second, through sensors, software, and connectivity,
	%you now have the capability to collect data, process information, and learn in
	%ways that would have been difficult if not impossible in the industrial world
	
	
	
	
	
	
	
	
%-------------------------------------------------------------------------------------------
	
	\section{Requirements to become a fast mover} \label{sec:ReqFastMovers}
	
	%Argue for what leadership capabilities and digital capabilities an organization should prioritize building in order to become a fast mover. 
	
	
	
	
	
	
	\newpage
	\addcontentsline{toc}{section}{References}
	\printbibliography
	
	
	
	
	
	
\end{document}