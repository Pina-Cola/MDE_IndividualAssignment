\documentclass[a4]{scrartcl}

% \usepackage[ngerman]{babel}
\usepackage[utf8]{inputenc}
\usepackage{mathtools}
\usepackage{amsmath}
\usepackage{amssymb}
\usepackage{geometry}
\usepackage{scrlayer-scrpage}
\pagestyle{scrheadings}
\clearscrheadfoot

\setlength{\parindent}{0em}

\setlength{\parskip}{1.3ex}

\usepackage[onehalfspacing]{setspace}


\geometry{
	paper=a4paper, % Change to letterpaper for US letter
	top=2cm, % Top margin
	bottom=1.5cm, % Bottom margin
	left=2cm, % Left margin
	right=3cm, % Right margin
	%showframe, % Uncomment to show how the type block is set on the page
}

\usepackage[backend=biber, maxbibnames=99]{biblatex}
\addbibresource{references.bib}





\usepackage[framemethod=TikZ]{mdframed}

% Style %
\mdfdefinestyle{enviStyle}{
	innertopmargin = 10pt,
	linewidth      = 1pt,
	frametitlerule = true,
	roundcorner    = 2pt%
}





\newenvironment{CountingDefinition}[2][]{%
	\ifstrempty{#1}%
	{\mdfsetup{%
			frametitle={{\strut ~}}}
	}%
	{\mdfsetup{%
			frametitle={{\strut ~#1}}}%
	}%
	\mdfsetup{
		nobreak                   = true,
		linecolor                 = gray,
		frametitlebackgroundcolor = gray!50,
		style                     = enviStyle
	}
	\begin{mdframed}[]\relax%
		\label{#2}}{\end{mdframed}}









%\ohead{\\
%	Pina Kolling\\
%	piko0011}

\begin{document}
	
	\begin{titlepage}
		\centering
		{\scshape\LARGE Umeå University \par}
		\vspace{1cm}
		{\scshape\Large Managing the Digital Enterprise \par }
		\vspace{1.5cm}
		{\huge\bfseries  Individual Assignment 1 \par}
		\vspace{2cm}
		{\Large\itshape Pina Kolling\par}
		\vfill
		supervised by \par 
		\vspace{1cm}
		Dr. Daniel \textsc{Skog} \par 
		and \par 
		M. Sc. Ramy \textsc{Shenouda} 
		
		\vfill
		
		% Bottom of the page
		{\large \today\par}
	\end{titlepage}
	
	\setcounter{page}{1}
	
	\tableofcontents
	
	\newpage
	
	% Assignment 1 (For the grade Pass)
	%As digitalization increasingly permeates all levels of society, the conditions, tools and challenges for contemporary organizing are changing. The course literature provides different (but interconnected) perspectives, explanations and analyses of the changing nature of organizing. Your task is to: 
	
	%Use the concepts of scale, scope, and speed to explain how business and organizing conditions are changing for companies as their environments are increasingly digitalized.
	%Explain the concept of "fast movers" and use the concept of the nexus of scale, scope and speed to explain how organizations can become fast movers.
	%Argue for what leadership capabilities and digital capabilities an organization should prioritize building in order to become a fast mover. 
	
	
	%In order to pass, the student needs to: 
	%Present the concepts of scale, scope, speed and their dynamics in a way that corresponds with that of the course literature.  
	%Use the concepts of scale, scope, and speed to clearly demonstrate how conditions for business and organizing change in digitalized envrionments. 
	%Present the concept of fast mover in a way that corresponds with that of the course literature. 
	%Illustrate how the concept of fast movers logically follows from the conditions offered by the nexus of scale, scope, and speed. 
	%Present clear and grounded arguments to which digital and leadership capabilities an organization should focus on and why when it wants to become a fast mover.
	%Instructions:
	%Submit your answer as a .pdf file
	%The answer to the assignment has a limit of 2500 words
	%Read the questions carefully and make sure that your answers meet the stated grading criteria.
	%Be ambitious in terms of language, indication of sources, and structure. The university library has useful tips to assist you with this at https://writingguide.se/
	%Please note that your arguments must be grounded in the course literature (unless otherwise is specified by the question). You are, however, encouraged to make use of additional sources.
	%This is an individual exam and any attempts to plagiarize will be identified by a plagiarism software. Plagiarism in any form will be reported and followed by disciplinary action.
	%For this assignment, students can obtain the grade Pass or Fail.
	%Your results will be communicated no later than 15 working days after assignment deadline.
	%If a student does not obtain Pass after the first attempt, a re-examn will be offered, however not earlier than 10 working days after the grades from the first round have been communicated.












%-------------------------------------------------------------------------------------------
	\section{Effects of digitalization} \label{sec:EffectsOfDigitalization}
	
	%Use the concepts of scale, scope, and speed to explain how business and organizing conditions are changing for companies as their environments are increasingly digitalized.
	
	The digitalization of business and organization has a big impact on companies, reshaping the way they operate and compete. These changes can be analyzed through the lens of \textit{Scale}, \textit{Scope}, and \textit{Speed}. 
	
	\begin{CountingDefinition}[Definitions]{def:defdef}
		
		\begin{tabular}{lp{13.5cm}}
			
			\textbf{Scale} & describes the rate at which products are created.~\cite{scalescopespeedslides, digitalmatrix} \\
			\textbf{Scope} & describes the investment in customer relationships or physical distribution.~\cite{scalescopespeedslides, digitalmatrix} \\
			\textbf{Speed} & describes a way in which existing capabilities are improved and new capabilities are developed.~\cite{scalescopespeedslides, digitalmatrix} \\
			
		\end{tabular}
	
				
	\end{CountingDefinition}


	\textit{Scale}, \textit{Scope} and \textit{Speed} are also referred to as the three dimensions of digital business.~\cite{scalescopespeedslides, digitalmatrix}









%----------------------------------------------------------------------------------------------
\subsection{Scale} \label{subsec:Scale}
 
\vspace*{1.3em}
\begin{CountingDefinition}[Scale]{def:scale}
	Scale describes the rate at which products are created. \cite{scalescopespeedslides, digitalmatrix}
\end{CountingDefinition}


Companies extended their \textit{Scale} in the industrial era by increasing sales of products. This was limited by physical attributes such as reaching enough costumers. Digitalization enables companies to reach a global market more easily, because the costs of communication, transportation and distribution were decreased drastically. \cite{digitalgoods, digitalmatrix, wiwi}


Another aspects of the change of a company's \textit{Scope} in the context of digitalization is the existence of digital goods. Goods can be seperated into the categories of material goods, services and digital goods. Digital goods can easily and cheap be reproduced and distributed. A physical product (material good) has to be produced more often, which generates more costs for material, transport, logistics and production in general. A digital product can often be reproduced and distributed on a bigger scale easily, after the product was created once. Only maintenance aspects like potential server costs can increase -- the initial development costs of the product will not change. \cite{digitalgoods, wiwi}


But digital goods introduce new challenges like a high rate of piracy, because the digital goods can easily be reproduced and distributed by the consumer, too. \cite{digitalgoods}







%----------------------------------------------------------------------------------------------
\subsection{Scope} \label{subsec:Scope}

\begin{CountingDefinition}[Scope]{def:scope}
	Scope describes the investment in customer relationships or	physical distribution. \cite{scalescopespeedslides, digitalmatrix}
\end{CountingDefinition}









%----------------------------------------------------------------------------------------------
\subsection{Speed} \label{subsec:Speed}

\begin{CountingDefinition}[Speed]{def:speed}
	Speed describes a way in which existing capabilities are improved and new capabilities are developed. \cite{scalescopespeedslides, digitalmatrix}
\end{CountingDefinition}









	
	
	
	TODO
	
	
	\textbf{Scale:}
	\begin{itemize}
		\item Digitalization enables companies to reach a global scale more easily.
		\item Scalability in terms of infrastructure and resources has also improved.
		\item Data-driven decision-making at scale is now possible.
	\end{itemize}
	
	
	\textbf{Scope:}
	\begin{itemize}
		\item Digitalization has broadened the scope of business activities.
		\item The scope of competition has expanded as well.
		\item Business ecosystems and partnerships have become more important.
	\end{itemize}
	
	
	\textbf{Speed:}
	\begin{itemize}
		\item Digitalization has accelerated the pace of business operations.
		\item Product development cycles have shortened.
		\item Customer expectations for speed have increased.
	\end{itemize}
	
	Overall, digitalization has transformed the business landscape by increasing the potential for scale, expanding the scope of business activities, and accelerating the speed of operations. However, companies must also adapt to these changes by developing digital capabilities, embracing innovation, and continuously monitoring and adjusting their strategies to thrive in the digital era. Failure to do so can leave them at a competitive disadvantage in an increasingly digitalized world.
	
	
	
	
	
	
	
	
%-------------------------------------------------------------------------------------------	
	
	\section{Fast movers} \label{sec:FastMovers}
	
	%Explain the concept of "fast movers" and use the concept of the nexus of scale, scope and speed to explain how organizations can become fast movers.
	
	
	
	
	
	
	
	
	
	
	
%-------------------------------------------------------------------------------------------
	
	\section{Requirements to become a fast mover} \label{sec:ReqFastMovers}
	
	%Argue for what leadership capabilities and digital capabilities an organization should prioritize building in order to become a fast mover. 
	
	
	\cite{leadingdigital}
	
	
	
	
	\newpage
	\addcontentsline{toc}{section}{References}
	\printbibliography
	
	
	
	
	
	
\end{document}