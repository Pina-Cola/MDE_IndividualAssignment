\documentclass[a4]{scrartcl}

% \usepackage[ngerman]{babel}
\usepackage[utf8]{inputenc}
\usepackage{mathtools}
\usepackage{amsmath}
\usepackage{amssymb}
\usepackage{geometry}
\usepackage{scrlayer-scrpage}
\pagestyle{scrheadings}
\usepackage{tablefootnote}
\usepackage[dvipsnames]{xcolor}
% \clearscrheadfoot

\setlength{\parindent}{0em}

\setlength{\parskip}{1.3ex}

\usepackage[onehalfspacing]{setspace}


\clubpenalty = 10000
\widowpenalty = 10000
\displaywidowpenalty = 10000


\geometry{
	paper=a4paper, % Change to letterpaper for US letter
	top=3cm, % Top margin
	bottom=3cm, % Bottom margin
	left=2cm, % Left margin
	right=3cm, % Right margin
	%showframe, % Uncomment to show how the type block is set on the page
}

\usepackage[backend=biber, maxbibnames=99]{biblatex}
\addbibresource{references.bib}



\usepackage[framemethod=TikZ]{mdframed}

% Style %
\mdfdefinestyle{enviStyle}{
	innertopmargin = 10pt,
	linewidth      = 1pt,
	frametitlerule = true,
	roundcorner    = 2pt%
}


\usepackage{sectsty}
\sectionfont{\color{BrickRed}}
\subsectionfont{\color{BrickRed}}

\newenvironment{CountingDefinition}[2][]{%
	\ifstrempty{#1}%
	{\mdfsetup{%
			frametitle={{\strut ~}}}
	}%
	{\mdfsetup{%
			frametitle={{\strut ~#1}}}%
	}%
	\mdfsetup{
		nobreak                   = true,
		linecolor                 = BrickRed,
		frametitlebackgroundcolor = BrickRed!50,
		style                     = enviStyle
	}
	\begin{mdframed}[]\relax%
		\label{#2}}{\end{mdframed}}


\renewcommand{\labelitemi}{$\textcolor{BrickRed}{\bullet}$}
\renewcommand{\labelitemii}{$\textcolor{BrickRed}{\cdot}$}
\renewcommand{\labelitemiii}{$\textcolor{BrickRed}{\diamond}$}
\renewcommand{\labelitemiv}{$\textcolor{BrickRed}{\ast}$}






%\ohead{\\
%	Pina Kolling\\
%	piko0011}

\begin{document}
	
	\begin{titlepage}
		\centering
		{\scshape\LARGE Umeå University \par}
		\vspace{1cm}
		{\scshape\Large Managing the Digital Enterprise \par }
		\vspace{1.5cm}
		{\huge\bfseries  {\color{BrickRed}Individual Assignment 2} \par}
		\vspace{2cm}
		{\Large\itshape Pina Kolling\par}
		\vfill
		supervised by \par 
		\vspace{1cm}
		Dr. Daniel \textsc{Skog} \par 
		and \par 
		M. Sc. Ramy \textsc{Shenouda} 
		
		\vfill
		
		% Bottom of the page
		{\large \today\par}
	\end{titlepage}
	
	\setcounter{page}{1}
	
	\begin{doublespace}
		\tableofcontents
	\end{doublespace}

	
	\newpage


%Digital transformation includes more than just the implementation and use of new digital technology in an organization. It is often instigated and fueled by external influences and resources, it is driven and maintained by humans in the organization, and it ultimately leads to changes that extend beyond process improvements. Taking the perspective of incumbent companies (companies that are already well established and successful), and by grounding your answer in the course literature and other relevant research, your task is to:

%1. Describe three factors related to the digitalization of the business and technology environments around incumbent companies and explain how and why they are likely to motivate or force them to engage in digital transformation.
%2. Identify and explain key challenges that are particularly likely to manifest in the digital transformation of established and historically successful companies.
%3. Apply relevant concepts from the course literature to discuss if and how incumbent organizations can avoid, mitigate, or manage these challenges.










%1. Describe three factors related to the digitalization of the business and technology environments around incumbent companies and explain how and why they are likely to motivate or force them to engage in digital transformation.
%-------------------------------------------------------------------------------------------
\section{Digital Transformation in Incumbent Companies} \label{sec:Sec1}

Digitalization is a defining force in today's business landscape, reshaping company's operations and competitive strategies across various industries. They are changing constantly and rapidly to adapt to shifting market conditions and evolving circumstances. 
This transformation has also an impact on incumbent companies. \cite{masterthesis, ZHANG}

\begin{CountingDefinition}[Incumbent company]{def:IncumbentCompany}
	
	Incumbent company refers to a well-established entity that has a significant presence and history within a particular industry or market. These companies have been in operation for a long period, often for many years or decades, and have typically achieved a level of market leadership, brand recognition, and customer base.  \cite{digitalmatrix, leadingdigital}
	
\end{CountingDefinition}


The adaptation of incumbent companies to the digital progress is often a necessity to remain relevant and competitive in a world where technology and customer expectations are constantly changing. \cite{digitalmatrix, leadingdigital}

The digital transformation alters whole businesses and their strategies by focussing more on the customer. Incumbent companies are particularly facing challenges in changing and adapting quickly, because their organizational structures and strategies have already been executed without change for many years. \cite{masterthesis}

Three key factors related to the digitalization of the business and technology environments around incumbent companies and their motivations for engaging in digital transformation are described below. 





%-------------------------------------------------------------------------------------------	
\subsection{Customer Expectations} \label{subsec:CustomerExpectations}

Changing customer expectations due to digitalization are an aspect of the evolving business landscape.
As technology advances, consumers have come to expect seamless and personalized experiences when interacting with businesses, including e-commerce, mobile apps or customer service. Consumers demand easy access, instant responses, and tailored solutions.  \cite{CustomerExpectations, masterthesis,digitalmatrix, leadingdigital}

In the following are listed some aspects of the change in customer expectations due to digitalization:

\begin{itemize}
	\item \textbf{Personalization and Customization}: 
	
	Digitalization enables businesses to collect and analyse data about their customers. As a result, customers expect a more personalized and tailored experience. They even expect that businesses will use their data to provide recommendations, promotions, and content that align with their preferences and behaviours.
	
	\item TODO
	
	\item **Convenience and Accessibility**: The convenience of digital channels, such as e-commerce websites, mobile apps, and online services, has set a new standard. Customers expect quick and easy access to products, services, and information. They want to interact with businesses on their own terms, whether it's through self-service options or instant access to customer support.
	
	\item **Real-Time Interactions**: Digitalization has enabled real-time communication, and customers now expect rapid responses and solutions. This applies not only to customer support but also to the way businesses handle transactions, feedback, and engagement. Social media platforms and instant messaging services have created an expectation of immediate communication.
	
	\item **Omnichannel Experiences**: Customers often engage with businesses through multiple channels, and they expect a seamless experience across these channels. For example, if a customer starts an interaction on a website and later switches to a mobile app or contacts customer support, they expect consistency and continuity in their journey.
	
	\item **Transparency and Trust**: Digitalization has increased transparency, making it easier for customers to access information about products, prices, and company practices. Customers now expect honesty and integrity from businesses. Any discrepancy between a company's digital image and its real-world actions can lead to a loss of trust.
	
	\item **User-Friendly Design**: The design and usability of digital interfaces have become crucial. Customers expect intuitive, user-friendly websites and applications. A poor user experience can drive customers away, even if the product or service is valuable.
	
	\item **Reviews and Social Proof**: Customers rely on reviews and social proof more than ever before when making purchasing decisions. They expect companies to provide a platform for reviews and actively respond to feedback.
	
	\item **Security and Privacy**: With the increased sharing of personal data, customers are more concerned about their privacy and data security. They expect businesses to take robust measures to protect their information and are likely to be more loyal to companies that prioritize data security.
	
	\item **Sustainability and Social Responsibility**: Digitalization has amplified awareness of environmental and social issues. Customers expect businesses to show a commitment to sustainability and social responsibility, and they may choose to support companies that align with their values.
\end{itemize}






TODO write more (?)	
	
		
Incumbent companies must adapt to these changing expectations to remain competitive. Traditional businesses are forced to establish an online presence and enhance their digital channels to engage with  customers effectively. Companies that fail to meet these expectations risk losing market share. \cite{masterthesis,digitalmatrix, leadingdigital}

	
	
	
%-------------------------------------------------------------------------------------------	
\subsection{New Technologies and Competitors} \label{subsec:NewTechnologiesandCompetitors}

	TODO

	Motivation: Startups and tech-native companies often leverage cutting-edge technologies to disrupt established industries. Innovations like artificial intelligence, blockchain, Internet of Things (IoT), and data analytics are reshaping industries and enabling new business models.
	
	
	
	Impact: Incumbent companies face the risk of being left behind by competitors who leverage these technologies for cost-efficiency, improved customer experiences, or entirely new offerings. To remain relevant and competitive, they must adopt digital strategies, embrace innovation, and incorporate these emerging technologies into their operations.
	
	
%-------------------------------------------------------------------------------------------	
\subsection{Data-Driven Decision-Making} \label{subsec:DataDrivenDecisionMaking}

	TODO

	Motivation: The digital environment generates vast amounts of data that can be harnessed for insights, predictions, and informed decision-making. Companies that effectively use data analytics gain a competitive edge by understanding customer behavior, optimizing operations, and predicting trends.
	
	
	Impact: Incumbent companies may have large datasets accumulated over the years. To unlock the potential of this data, they need to invest in data analytics tools and expertise. Data-driven decision-making enables them to enhance efficiency, personalize customer experiences, and uncover new revenue opportunities.
	
	
	
	In summary, changing consumer expectations, competitive disruption driven by emerging technologies, and the potential of data-driven decision-making motivate incumbent companies to engage in digital transformation. Adapting to these factors is crucial for their survival and long-term success in the digital age. Companies that resist these changes risk becoming obsolete, losing market share, and missing out on opportunities for growth and innovation.
	
	








%2. Identify and explain key challenges that are particularly likely to manifest in the digital transformation of established and historically successful companies.
%-------------------------------------------------------------------------------------------
\section{Key challenges for established companies} \label{sec:Sec2}












%3. Apply relevant concepts from the course literature to discuss if and how incumbent organizations can avoid, mitigate, or manage these challenges.
%-------------------------------------------------------------------------------------------
\section{Strategies for overcoming challenges} \label{sec:Sec3}











%-------------------------------------------------------------------------------------------

	
\newpage
\addcontentsline{toc}{section}{References}
\begin{spacing}{0.9}
	\printbibliography
\end{spacing}


	
	
	
	
	
	
\end{document}