\documentclass[a4]{scrartcl}

% \usepackage[ngerman]{babel}
\usepackage[utf8]{inputenc}
\usepackage{mathtools}
\usepackage{amsmath}
\usepackage{amssymb}
\usepackage{geometry}
\usepackage{scrlayer-scrpage}
\pagestyle{scrheadings}
\usepackage{tablefootnote}
\usepackage[dvipsnames]{xcolor}
% \clearscrheadfoot


\setlength{\parindent}{0em}

\setlength{\parskip}{1.3ex}

\usepackage[onehalfspacing]{setspace}


\usepackage{hyperref}
\hypersetup{
	colorlinks=true,
	linkcolor=black,
	filecolor=black,      
	urlcolor=BrickRed,
	citecolor=black
}


\clubpenalty = 10000
\widowpenalty = 10000
\displaywidowpenalty = 10000


\geometry{
	paper=a4paper, % Change to letterpaper for US letter
	top=3cm, % Top margin
	bottom=3cm, % Bottom margin
	left=2cm, % Left margin
	right=3cm, % Right margin
	%showframe, % Uncomment to show how the type block is set on the page
}

\usepackage[backend=biber, maxbibnames=99]{biblatex}
\addbibresource{references.bib}



\usepackage[framemethod=TikZ]{mdframed}

% Style %
\mdfdefinestyle{enviStyle}{
	innertopmargin = 10pt,
	linewidth      = 1pt,
	frametitlerule = true,
	roundcorner    = 2pt%
}


\usepackage{sectsty}
\sectionfont{\color{BrickRed}}
\subsectionfont{\color{BrickRed}}

\newenvironment{CountingDefinition}[2][]{%
	\ifstrempty{#1}%
	{\mdfsetup{%
			frametitle={{\strut ~}}}
	}%
	{\mdfsetup{%
			frametitle={{\strut ~#1}}}%
	}%
	\mdfsetup{
		nobreak                   = true,
		linecolor                 = BrickRed,
		frametitlebackgroundcolor = BrickRed!50,
		style                     = enviStyle
	}
	\begin{mdframed}[]\relax%
		\label{#2}}{\end{mdframed}}


\renewcommand{\labelitemi}{$\textcolor{BrickRed}{\bullet}$}
\renewcommand{\labelitemii}{$\textcolor{BrickRed}{\cdot}$}
\renewcommand{\labelitemiii}{$\textcolor{BrickRed}{\diamond}$}
\renewcommand{\labelitemiv}{$\textcolor{BrickRed}{\ast}$}






%\ohead{\\
%	Pina Kolling\\
%	piko0011}

\begin{document}
	
	\begin{titlepage}
		\centering
		{\scshape\LARGE Umeå University \par}
		\vspace{1cm}
		{\scshape\Large Managing the Digital Enterprise \par }
		\vspace{1.5cm}
		{\huge\bfseries  {\color{BrickRed}Individual Assignment 2} \par}
		\vspace{2cm}
		{\Large\itshape Pina Kolling\par}
		\vfill
		supervised by \par 
		\vspace{1cm}
		Dr. Daniel \textsc{Skog} \par 
		and \par 
		M. Sc. Ramy \textsc{Shenouda} 
		
		\vfill
		
		% Bottom of the page
		{\large \today\par}
	\end{titlepage}
	
	\setcounter{page}{1}
	
	\begin{doublespace}
		\tableofcontents
	\end{doublespace}

	
	\newpage


%Digital transformation includes more than just the implementation and use of new digital technology in an organization. It is often instigated and fueled by external influences and resources, it is driven and maintained by humans in the organization, and it ultimately leads to changes that extend beyond process improvements. Taking the perspective of incumbent companies (companies that are already well established and successful), and by grounding your answer in the course literature and other relevant research, your task is to:

%1. Describe three factors related to the digitalization of the business and technology environments around incumbent companies and explain how and why they are likely to motivate or force them to engage in digital transformation.
%2. Identify and explain key challenges that are particularly likely to manifest in the digital transformation of established and historically successful companies.
%3. Apply relevant concepts from the course literature to discuss if and how incumbent organizations can avoid, mitigate, or manage these challenges.










%1. Describe three factors related to the digitalization of the business and technology environments around incumbent companies and explain how and why they are likely to motivate or force them to engage in digital transformation.
%-------------------------------------------------------------------------------------------
\section{Digital Transformation in Incumbent Companies} \label{sec:Sec1}

Digitalization is a defining force in today's business landscape, reshaping company's operations and competitive strategies across various industries. They are changing constantly and rapidly to adapt to shifting market conditions and evolving circumstances. 
This transformation has also an impact on incumbent companies. \cite{masterthesis, ZHANG}

\begin{CountingDefinition}[Incumbent company]{def:IncumbentCompany}
	
	Incumbent company refers to a well-established entity that has a significant presence and history within a particular industry or market. These companies have been in operation for a long period, often for many years or decades, and have typically achieved a level of market leadership, brand recognition, and customer base.  \cite{digitalmatrix, leadingdigital}
	
\end{CountingDefinition}


The adaptation of incumbent companies to the digital progress is often a necessity to remain relevant and competitive in a world where technology and customer expectations are constantly changing. \cite{digitalmatrix, leadingdigital}

The digital transformation alters whole businesses and their strategies by focussing more on the customer. Incumbent companies are particularly facing challenges in changing and adapting quickly, because their organizational structures and strategies have already been executed without change for many years. \cite{masterthesis}

Three key factors related to the digitalization of the business and technology environments around incumbent companies and their motivations for engaging in digital transformation are described below. 





%-------------------------------------------------------------------------------------------	
\subsection{Customer Expectations} \label{subsec:CustomerExpectations}

Changing customer expectations due to digitalization are an aspect of the evolving business landscape.
As technology advances, consumers have come to expect seamless and personalized experiences when interacting with businesses, including e-commerce, mobile apps, personal recommendations or customer service. Consumers demand easy access, instant responses, and tailored solutions.~\cite{CustomerExpectations, socialmedia, masterthesis,digitalmatrix, leadingdigital}

In the following are listed some aspects of the change in customer expectations due to digitalization:


\begin{itemize}
	
	
	
	\item \textbf{Personalization and Customization}: 
	
	Digitalization enables businesses to collect and analyse data about their customers. As a result, customers expect a more personalized and tailored experience. They even expect that businesses will use their data to provide recommendations, promotions, and content that align with their preferences and behaviours.~\cite{socialmedia, masterthesis, digitalmatrix, leadingdigital}
	
	One example for a company that has a tight connection with their customers wishes is the active wear brand Popflex\footnote{The brand can be found at \url{https://www.popflexactive.com/} and the designer of the clothes is really strong at social media marketing, for example at \url{https://www.youtube.com/@blogilates}.}. They redesign their clothes with the feedback and wishes of their customers, to make it fit on different body types and be as practical as possible.
	
	
	\item \textbf{Convenience and Accessibility}: 
	
	The convenience of digital channels, such as e-commerce websites, mobile apps, and online services, are the new standard. Customers expect quick and easy access to products, services, and information.~\cite{masterthesis, leadingdigital}
	
	
	
	\item \textbf{Customer Support and Real-Time Interactions}: 
	
	The internet has enabled real-time communication, which lead customers to expect quick responses and solutions. This includes not only customer support but also the way businesses handle transactions and other processes. Social media platforms and instant messaging services have created an expectation of immediate communication.~\cite{socialmedia, digitalmatrix, leadingdigital}
	
	
	\item \textbf{Transparency and Trust}: 
	
	Digitalization has increased the transparency of company practices, making it easier for customers to access information about the products, prices and processes of a company. Customers now expect honesty and integrity from businesses. Any discrepancy between a company's digital image and its real-world actions can lead to a loss of trust.
	Customers even expect businesses to show a commitment to sustainability and social responsibility.~\cite{trust}
	
	
	\item \textbf{User-Friendly Design and Social Media Marketing}: 
	
	The design and usability of digital interfaces and the digital marketing on social media have become crucial. Customers expect intuitive, user-friendly websites and applications and suitable social media marketing, including product placements, promotions or crowd sourcing. A poor user experience can drive customers away, even if the product or service is valuable.~\cite{socialmedia}
	In addition to this, customers rely on reviews and forums and social proof when making purchasing decisions. They might even expect companies to provide a platform for reviews and actively respond to feedback.~\cite{socialmedia}
	
	\item \textbf{Security and Privacy}: 
	
	With the increased sharing and collecting of data, customers are more concerned about their privacy and data security. They expect businesses to take robust measures to protect their information and to be secure against cyber attacks. \cite{masterthesis, cybersecurity}

\end{itemize}
	
		
Incumbent companies must adapt to these changing expectations to remain competitive. Traditional businesses are forced to establish an online presence and enhance their digital channels to engage with  customers effectively. Companies that fail to meet these expectations risk losing relevance on the market.~\cite{socialmedia, masterthesis, digitalmatrix, leadingdigital}

	
	

	
	
%-------------------------------------------------------------------------------------------	
\subsection{Data-Driven Decision-Making} \label{subsec:DataDrivenDecisionMaking}

	The digital environment enables companies the option to collect big amounts of data that can be analysed for insights, predictions, and informed decision-making. Companies that effectively use data analytics can have a competitive advantage by understanding customer behaviour, optimizing their operations and predicting trends.~\cite{DDDM, masterthesis}
	
	In the following are listed some advantages and applications of data-driven decision-making:
	
	\begin{itemize}
				
		
		\item \textbf{Personalization and Targeting}: 
		
		Digitalization and the information that was gained by collected data allows personalized and targeted marketing and product recommendations. A certain level of personalization enhances the customer experience and can increase sales.~\cite{socialmedia, DDDM}
		
		\item \textbf{Operational Efficiency}: 
		
		Data-driven decision-making is not only applied to improve the customer experience. Businesses can optimize their internal processes and operations by analyzing data to identify bottlenecks, inefficiencies, and areas for improvement. This can result in cost decreasing and increased productivity.~\cite{DDDM}
		
		\item \textbf{Predictive Analytics}: 
		
		Digitalization enables companies to use analytics and predict market trends and customer preferences.~\cite{DDDM, masterthesis, leadingdigital}

		
		\item \textbf{Market Responsiveness}: 
		
		Today's markets are changing rapidly. Data-driven decision-making enables companies to respond to these changes more effectively or even in real-time. They can quickly identify trends and adapt their strategies to new market demands.~\cite{DDDM, digitalmatrix, leadingdigital}
		
		
	\end{itemize}
	
	In this digitalized market, data-driven decision-making is not just a valuable tool; it's a strategic imperative for businesses looking to thrive in an environment characterized by constant change and intense competition.
	
	Organizations that effectively harness data-driven decision-making gain a significant competitive advantage. They can make informed decisions based on concrete evidence, leading to more successful product launches, targeted marketing campaigns, and optimized supply chains.
	

	
	
	Impact: Incumbent companies may have large datasets accumulated over the years. To unlock the potential of this data, they need to invest in data analytics tools and expertise. Data-driven decision-making enables them to enhance efficiency, personalize customer experiences, and uncover new revenue opportunities.
	



%-------------------------------------------------------------------------------------------	
\subsection{New Technologies and Competitors} \label{subsec:NewTechnologiesandCompetitors}

TODO

Motivation: Startups and tech-native companies often leverage cutting-edge technologies to disrupt established industries. Innovations like artificial intelligence, blockchain, Internet of Things (IoT), and data analytics are reshaping industries and enabling new business models.



Impact: Incumbent companies face the risk of being left behind by competitors who leverage these technologies for cost-efficiency, improved customer experiences, or entirely new offerings. To remain relevant and competitive, they must adopt digital strategies, embrace innovation, and incorporate these emerging technologies into their operations.
	
	
	
	
	
	In summary, changing consumer expectations, competitive disruption driven by emerging technologies, and the potential of data-driven decision-making motivate incumbent companies to engage in digital transformation. Adapting to these factors is crucial for their survival and long-term success in the digital age. Companies that resist these changes risk becoming obsolete, losing market share, and missing out on opportunities for growth and innovation.
	
	








%2. Identify and explain key challenges that are particularly likely to manifest in the digital transformation of established and historically successful companies.
%-------------------------------------------------------------------------------------------
\section{Key challenges for established companies} \label{sec:Sec2}












%3. Apply relevant concepts from the course literature to discuss if and how incumbent organizations can avoid, mitigate, or manage these challenges.
%-------------------------------------------------------------------------------------------
\section{Strategies for overcoming challenges} \label{sec:Sec3}











%-------------------------------------------------------------------------------------------

	
\newpage
\addcontentsline{toc}{section}{References}
\begin{spacing}{0.9}
	\printbibliography
\end{spacing}


	
	
	
	
	
	
\end{document}