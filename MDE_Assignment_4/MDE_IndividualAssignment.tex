\documentclass[a4]{scrartcl}

% \usepackage[ngerman]{babel}
\usepackage[utf8]{inputenc}
\usepackage{mathtools}
\usepackage{amsmath}
\usepackage{amssymb}
\usepackage{geometry}
\usepackage{scrlayer-scrpage}
\pagestyle{scrheadings}
\usepackage{tablefootnote}
\usepackage[dvipsnames]{xcolor}
% \clearscrheadfoot

\setlength{\parindent}{0em}

\setlength{\parskip}{1.3ex}

\usepackage[onehalfspacing]{setspace}


\clubpenalty = 10000
\widowpenalty = 10000
\displaywidowpenalty = 10000

\usepackage{hyperref}
\hypersetup{
	colorlinks=true,
	linkcolor=black,
	filecolor=black,      
	urlcolor=MidnightBlue,
	citecolor=black
}


\geometry{
	paper=a4paper, % Change to letterpaper for US letter
	top=3cm, % Top margin
	bottom=3cm, % Bottom margin
	left=2cm, % Left margin
	right=3cm, % Right margin
	%showframe, % Uncomment to show how the type block is set on the page
}

\usepackage[backend=biber, maxbibnames=99]{biblatex}
\addbibresource{references.bib}



\usepackage[framemethod=TikZ]{mdframed}

% Style %
\mdfdefinestyle{enviStyle}{
	innertopmargin = 10pt,
	linewidth      = 1pt,
	frametitlerule = true,
	roundcorner    = 2pt%
}


\usepackage{sectsty}
\sectionfont{\color{MidnightBlue}}
\subsectionfont{\color{MidnightBlue}}



\newenvironment{CountingDefinition}[2][]{%
	\ifstrempty{#1}%
	{\mdfsetup{%
			frametitle={{\strut ~}}}
	}%
	{\mdfsetup{%
			frametitle={{\strut ~#1}}}%
	}%
	\mdfsetup{
		nobreak                   = true,
		linecolor                 = MidnightBlue,
		frametitlebackgroundcolor = MidnightBlue!50,
		style                     = enviStyle
	}
	\begin{mdframed}[]\relax%
		\label{#2}}{\end{mdframed}}



\renewcommand{\labelitemi}{$\textcolor{MidnightBlue}{\bullet}$}
\renewcommand{\labelitemii}{$\textcolor{MidnightBlue}{\cdot}$}
\renewcommand{\labelitemiii}{$\textcolor{MidnightBlue}{\diamond}$}
\renewcommand{\labelitemiv}{$\textcolor{MidnightBlue}{\ast}$}





%\ohead{\\
%	Pina Kolling\\
%	piko0011}

\begin{document}
	
	\begin{titlepage}
		\centering
		{\scshape\LARGE Umeå University \par}
		\vspace{1cm}
		{\scshape\Large Managing the Digital Enterprise \par }
		\vspace{1.5cm}
		{\huge\bfseries   {\color{MidnightBlue}Individual Assignment 4} \par}
		\vspace{2cm}
		{\Large\itshape Pina Kolling\par}
		\vfill
		supervised by \par 
		\vspace{1cm}
		Dr. Daniel \textsc{Skog} \par 
		and \par 
		M. Sc. Ramy \textsc{Shenouda} 
		
		\vfill
		
		% Bottom of the page
		{\large \today\par}
	\end{titlepage}
	
	\setcounter{page}{1}
	
	\begin{doublespace}
		\tableofcontents
	\end{doublespace}

	
	\newpage

%The concept of digital transformation is understood differently by different people in both research and practice. Considering the amount of resources that are being invested in strategic digital transformation initiatives, it is of vital importance that managers of digital enterprises have a clear view of what it entails. In particular, it is of high importance that managers understand the extent of change that is involved for the organization. Your task in this assignment is to provide a synthesis of research that a manager of a digital enterprise can use to understand when a change stemming from technology emergence and use should be considered a digital transformation and not. Based on your synthesis, your task is also to provide research-grounded advice on what a manager could expect and plan for when initiating digital transformation in the organization. Specifically, your task is to:












% 1. Value, select and summarize research concepts, perspectives and arguments from at least three of the papers below concerning when a process should be considered to be a digital transformation and not. 
%-------------------------------------------------------------------------------------------
	\section{Definitions of Digital Transformation} \label{sec:Sec1}







%-------------------------------------------------------------------------------------------	
	\subsection{\textit{Digital doesn't have to be disruptive}} \label{disruptive}
	
	Nathan Furr and Andrew Shipilov described digital transformation in \textit{Digital doesn't have to be disruptive: the best results can come from adaptation rather than reinvention} as ``adapting an organization's strategy and structure to capture opportunities enabled by digital technology``~\cite[p. 96]{disruptive}.
	It was stated that it can be difficult for companies to create a plan on how to act and execute their digital transformation.
	
	\begin{itemize}
		\item main aspects: automation, virtualization, more targeted product and service customization, more informed decision making and machine-driven recommendations
		\item technology is applied at almost every company and in every step of their processes
		\item radical replacements are only sometimes necessary -- digital transformation means incremental steps to improve the processes
		\item challenge for digital transformation: find the best way to full fill goals using digital tools as helpers or to overcome previous challenges
		\item  get more efficient and user-friendly through digital tools
	\end{itemize}









%-------------------------------------------------------------------------------------------	
\subsection{\textit{Five myths about digital transformation}} \label{5myths}	
	
	Stephen J. Andriole in \textit{Five myths about digital transformation}. \cite{5myths}
	
	\begin{itemize}
		\item path to digital transformation is risky but it might lead to efficiency, innovation and competitiveness
		\item companies will fail to implement digital transformation unless it is well planned and executed
		\item collected five myths to make the reader aware of the risks and dangers of digital transformation
		\item digital transformation is hyped and not described as risky enough
		\item 1: ''not every company, process, or business model requires digital transformation'' \cite[p. 20]{5myths}
		\item 2: digital transformation does not necessarily use emerging or disruptive technologies
		\item 3: if the company is already going well, the transformation will not have a meaningful impact
		\item 4: disruptive transformation does usually not begin with the market leaders
		\item 5: executives do not necessarily want to transform digitally 
	\end{itemize}










%-------------------------------------------------------------------------------------------	
\subsection{\textit{IT-enabled business transformation}} \label{venkat}	

Nramanujam Venkatraman in \textit{IT-enabled business transformation: from automation to business scope redefinition}. \cite{venkat}


	\begin{itemize}
		\item IT has a distinctive role in shaping the future's business operations
		\item IT is a fundamental enabler in creating and maintaining shit
		\item 5 levels of IT enables business operations
		\item companies should estimate first the costs and efforts in comparison to the benefits and then move to higher levels
		\item 1: Localized Exploitation -- deployment of standard IT applications with minimal changes to the business processes
		\item 2: Internal Integration -- deployment of IT applications in the entire business process
		\item 3: Business Process Redesign -- renew processes to improve efficiency with IT applications
		\item 4: Business Network Redesign -- digital transformation not only within the organization but with partners or suppliers
		\item 5: Business Scope Redefinition -- redefine the market and the company's goals and potentially outsource tasks to third party companies
	\end{itemize}
	
	
	
	
	
	
	
%-------------------------------------------------------------------------------------------	
\subsection{\textit{Understanding digital transformation}} \label{vial}	
	
Gregory Vial in \textit{Understanding digital transformation: A review and a research agenda} \cite{vial}.

\begin{itemize}
	\item digital transformation consists out of 8 building blocks: digital technologies, disruption, strategic responses, value creation paths, structural changes, organizational barriers, positive and negative outcomes
	\item table with different definitions of digital transformation
	\item include graphics of the 8 step framework and describe
\end{itemize}
	











	





%-------------------------------------------------------------------------------------------	
\subsection{\textit{Digital Transformation versus IT-Enabled Transformation}} \label{DTOT}	

Lauri Wessel, Abayomi Baiyere, Roxana Ologeanu-Taddei, Jonghyuk Cha and Tina Blegind-Jensen in \textit{Unpacking the difference between digital transformation and IT-enabled organizational transformation}. \cite{DTOT}
	
	\begin{itemize}
		\item Focus on difference between digital transformation and IT-enabled transformation
		\item digital transformation can lead to a new organizational identity while IT-enabled organizational transformation is the enhancement of an existing organizational identity
	\end{itemize}
	














% 2. Analyze and synthesize the concepts, perspectives and arguments to individually argue for 3-4 key characteristics that can be used to determine whether a process should be considered to be a digital transformation and not. 
%-------------------------------------------------------------------------------------------
\section{Section 2} \label{sec:Sec2}

\begin{CountingDefinition}[Definitions]{def:defdef}
	
		
		Text

	
	
\end{CountingDefinition}











% 3. Use the 3-4 key characteristics to explain what scope and scale of change a manager should expect and plan for when initiating a digital transformation initiative and the most important things a manager needs to think about to manage it successfully
%-------------------------------------------------------------------------------------------
\section{Section 3} \label{sec:Sec3}

\begin{CountingDefinition}[Definitions]{def:defdef}
	

		
		Text

	
\end{CountingDefinition}













%-------------------------------------------------------------------------------------------
	
	\newpage
	\addcontentsline{toc}{section}{References}
	\begin{spacing}{0.9}
		\printbibliography
	\end{spacing}


	
	
	
	
	
	
\end{document}