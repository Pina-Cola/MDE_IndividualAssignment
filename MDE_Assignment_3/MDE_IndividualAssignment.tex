\documentclass[a4]{scrartcl}

% \usepackage[ngerman]{babel}
\usepackage[utf8]{inputenc}
\usepackage{mathtools}
\usepackage{amsmath}
\usepackage{amssymb}
\usepackage{geometry}
\usepackage{scrlayer-scrpage}
\pagestyle{scrheadings}
\usepackage{tablefootnote}
\usepackage[dvipsnames]{xcolor}
% \clearscrheadfoot

\setlength{\parindent}{0em}

\setlength{\parskip}{1.3ex}

\usepackage[onehalfspacing]{setspace}


\clubpenalty = 10000
\widowpenalty = 10000
\displaywidowpenalty = 10000

\usepackage{hyperref}
\hypersetup{
	colorlinks=true,
	linkcolor=black,
	filecolor=black,      
	urlcolor=BurntOrange,
	citecolor=black
}


\geometry{
	paper=a4paper, % Change to letterpaper for US letter
	top=3cm, % Top margin
	bottom=3cm, % Bottom margin
	left=2cm, % Left margin
	right=3cm, % Right margin
	%showframe, % Uncomment to show how the type block is set on the page
}

\usepackage[backend=biber, maxbibnames=99]{biblatex}
\addbibresource{references.bib}
\setcounter{biburllcpenalty}{7000}
\setcounter{biburlucpenalty}{8000}



\usepackage[framemethod=TikZ]{mdframed}

% Style %
\mdfdefinestyle{enviStyle}{
	innertopmargin = 10pt,
	linewidth      = 1pt,
	frametitlerule = true,
	roundcorner    = 2pt%
}


\usepackage{sectsty}
\sectionfont{\color{BurntOrange}}
\subsectionfont{\color{BurntOrange}}


\newenvironment{CountingDefinition}[2][]{%
	\ifstrempty{#1}%
	{\mdfsetup{%
			frametitle={{\strut ~}}}
	}%
	{\mdfsetup{%
			frametitle={{\strut ~#1}}}%
	}%
	\mdfsetup{
		nobreak                   = true,
		linecolor                 = BurntOrange,
		frametitlebackgroundcolor = BurntOrange!50,
		style                     = enviStyle
	}
	\begin{mdframed}[]\relax%
		\label{#2}}{\end{mdframed}}



\renewcommand{\labelitemi}{$\textcolor{BurntOrange}{\bullet}$}
\renewcommand{\labelitemii}{$\textcolor{BurntOrange}{\cdot}$}
\renewcommand{\labelitemiii}{$\textcolor{BurntOrange}{\diamond}$}
\renewcommand{\labelitemiv}{$\textcolor{BurntOrange}{\ast}$}





%\ohead{\\
%	Pina Kolling\\
%	piko0011}

\begin{document}
	
	\begin{titlepage}
		\centering
		{\scshape\LARGE Umeå University \par}
		\vspace{1cm}
		{\scshape\Large Managing the Digital Enterprise \par }
		\vspace{1.5cm}
		{\huge\bfseries   {\color{BurntOrange}Individual Assignment 3} \par}
		\vspace{2cm}
		{\Large\itshape Pina Kolling\par}
		\vfill
		supervised by \par 
		\vspace{1cm}
		Dr. Daniel \textsc{Skog} \par 
		and \par 
		M. Sc. Ramy \textsc{Shenouda} 
		
		\vfill
		
		% Bottom of the page
		{\large \today\par}
	\end{titlepage}
	
	\setcounter{page}{1}
	
	\begin{doublespace}
		\tableofcontents
	\end{doublespace}

	
	\newpage


% Two books in the course literature present and prescribe ways for organizations to manage digital transformation in useful ways. In order to make informed decisions as a manager of a digital enterprise of how to understand, evaluate, and potentially use different prescriptive statements, concepts, models or frameworks, a manager needs to be able to analyze and understand core assumptions underlying normative suggestions for action. For example, management literature tend to based on certain assumptions regarding who the reader is, what type of organization the person is in, what part of the world the organization operates, and prescribes advice accordingly. Your task is to:

%Identify core assumptions in Venkatraman (2017) and in Westerman et al. (2014).
%Discuss the consequences of these assumptions in terms of how digital transformation and approaches to managing it are portrayed in the two books.
%Use the results of 1 and 2, and your own example of an organization and context, to describe where, when and why the approach of either Venkatraman or Westerman et al, would not be suitable












%Identify core assumptions in Venkatraman (2017) and in Westerman et al. (2014).
%-------------------------------------------------------------------------------------------
	\section{Core assumptions in digital transformation literature} \label{sec:Sec1}
	
	In this Section, the core assumptions of Venkatraman in \textit{The digital matrix: new rules for business transformation through technology} \cite{leadingdigital} and Westerman, Bonnet and McAfee in \textit{Leading digital: Turning technology into business transformation} \cite{digitalmatrix} are presented.

%-------------------------------------------------------------------------------------------
% \subsection{Introduction of the authors} \label{subsec:authors}


	\begin{CountingDefinition}[Author of \textit{The digital matrix}]{def:defdef1}
		
		\begin{minipage}{0.3\linewidth}			
				\centering
				\includegraphics[width=0.65\textwidth]{images/VV.jpg}
				
				\tiny{Picture of Venkat Venkatraman\footnotemark} 
					
		\end{minipage}	\begin{minipage}{0.7\linewidth}
		
			Dr. Venkatraman holds a PhD from the University of Pittsburgh's (Katz Graduate School of Business, 1985). He specializes in the study of how established companies adapt to digital technologies. He published his knowledge in his book \textit{The Digital Matrix: New Rules for Business Transformation through Technology} in 2017. \cite{VV, digitalmatrix} 
		\end{minipage}

	
				
	\end{CountingDefinition}

	\footnotetext{Picture from \url{https://www.dukece.com/people/venkat-venkatraman/}}


	\begin{CountingDefinition}[Authors of \textit{Leading digital}]{def:defdef2}
		
		
		\begin{minipage}{0.3\linewidth}			
			\centering
			\includegraphics[width=0.5\textwidth]{images/GW.jpeg}
	
			\tiny{Picture of George Westermann\footnotemark} 
	
		\end{minipage}	\begin{minipage}{0.7\linewidth}
	
			George Westerman is a Senior Lecturer at MIT Sloan School of Management and Founder of the Global Opportunity Initiative. He has written award-winning books and conducted research on digital transformation. \cite{GW, leadingdigital}
			
		\end{minipage}



		\begin{minipage}{0.3\linewidth}			
			\centering
			\includegraphics[width=0.5\textwidth]{images/DB.jpg}
	
			\tiny{Picture of Didie Bonnet\footnotemark} 
	
		\end{minipage}	\begin{minipage}{0.7\linewidth}
	
			Dr. Didier Bonnet is specialized on digital transformation. He is a Professor at IMD Business School (Switzerland) and co-author of the book \textit{Leading digital}. He is featured on broadcasts like the BBC or CNN. \cite{DB2, DB1, leadingdigital}
			
		\end{minipage}


		\begin{minipage}{0.3\linewidth}			
			\centering
			\includegraphics[width=0.65\textwidth]{images/AM.jpg}
	
			\tiny{Picture of Andrew McAfee\footnotemark} 
	
		\end{minipage}	\begin{minipage}{0.7\linewidth}
	
			Andrew McAfee is a principal research scientist at MIT and co-founder of the MIT Initiative on the Digital Economy. He has written numeral books, including \textit{Race Against the Machine}, \textit{The Second Machine Age} and \textit{Leading digital}. \cite{AM2, AM3, AM1, leadingdigital}
			
		\end{minipage}



		
	\end{CountingDefinition}

	\footnotetext{Picture from \url{https://mitsloan.mit.edu/faculty/directory/george-f-westerman}}
	\footnotetext{Picture from \url{https://digitaltransformation2021.brightline.org/speakers/didier-bonnet/}}
	\footnotetext{Picture from \url{https://www.mckinsey.com/capabilities/strategy-and-corporate-finance/our-insights/the-strategy-and-corporate-finance-blog/leadership-rundown-is-technology-a-force-for-good}}


To effectively understand and use the literature and recommendations, it is important to critically analyse and understand the core assumptions that underlay their suggestions.
These assumptions might be the reader's position, the nature and market of the organization in question or its geographical context. 



%Notes (\textit{Leading digital} \cite{leadingdigital}):

%\begin{itemize}
%	\item Example brands: Nike and Asian Paints and Fashionistas
%	\item results in visible customer interactions and in internal operations, this is visible in the financials: Digital Masters are more profitable than their peers
%	\item digital transformation through strong top-down leadership
%	\item focused on top-down and leadership
%	\item 4 main aspects: shared transformative vision, strong governance, deep engagement and solid technology leadership
%	\item positve examples: Caesars, Codelco, P \& G, Pages Jaunes and Starbucks
%\end{itemize}


%Notes (\textit{Digital matrix} \cite{digitalmatrix}):

%\begin{itemize}
%	\item Examples: Nokia and BlackBerry being trapped and replaced by Apple and others
%	\item Example Microsoft
%	\item Example Marriott hotel chain
	
%\end{itemize}

%General notes:

%\begin{itemize}
%	\item American companies
%\end{itemize}

%-------------------------------------------------------------------------------------------
\subsection{Top-down approach} \label{subsec:topdown}


In the books, he execution of the digitalization was suggested with a top-down approach.
A top-down leadership approach in digital transformation can present challenges and lead to limitations. 
It often assumes that the employees are synchronized to a certain degree in terms of digital readiness and understanding. In reality, they might have different levels of digital understanding and readiness. In addition to this, top-down approaches can be slow in responding to challenges or changes, which can cause problems in the dynamic markets.
Depending on the culture of the company or the location of the headquarter, a top-down approach might not find acceptance and employees do not feel valued in their opinions.
The books assume a company and market environment, that is ready for digitalization and accepting a top-down approach to execute the changes. \cite{digitalmatrix, leadingdigital}







%-------------------------------------------------------------------------------------------
\subsection{Geographical context} \label{subsec:geo}

The geographical context in which a company operates is a critical factor. It has a big influence on the company's culture, employees, business environment, and technological infrastructure. 

To assess the pre assumptions that were made by the authors, the companies that were mentioned as an example were extracted and analysed. This extraction did not aim for completeness regarding finding every single example but there are enough data points to draw conclusions with. The extracted example companies and according headquater positions and industries are listed in Appendice \ref{a:LD} and \ref{b:DM}.







\begin{figure}[h!]
	\centering
	\includegraphics[width=0.9\textwidth]{images/LD_graph.png}
	\label{fig:LD_graph}
	\caption{Headquarters of the example companies in \textit{Leading digital} \cite{leadingdigital}}
\end{figure}


%**Cultural Nuances**: Different regions and countries have distinct cultural norms, values, and attitudes towards technology adoption. Understanding these nuances is vital for tailoring digital strategies that resonate with the local population.

%**Regulatory Environment**: Geographical location impacts the regulatory landscape. Compliance requirements, data protection laws, and industry-specific regulations can vary significantly from one region to another, necessitating localized strategies.

%**Market Dynamics**: The market dynamics in different geographical areas, including factors such as competition, customer behavior, and market saturation, can vary widely. Recognizing these distinctions is key to crafting effective digital transformation plans.

%**Access to Technology**: Availability and accessibility of technology infrastructure can differ based on location. Developing nations might face technology constraints, while developed countries may have advanced infrastructure, impacting the digitalization approach.

%**Economic Factors**: Economic conditions, such as GDP, inflation rates, and income levels, play a pivotal role in shaping digital strategies, especially regarding pricing, affordability, and market positioning.

%**Industrialization**: Whether a country is primarily industrial or service-oriented can influence the digital maturity of its businesses. Industrial nations might face unique challenges and opportunities in their transformation journey.



%In essence, the geographical context of a company, especially the location of its headquarters, introduces a myriad of factors that necessitate region-specific strategies for digital transformation. Ignoring these factors can hinder the effectiveness of digital initiatives. Therefore, the literature exemplifies the imperative need to consider the geographical context as an integral component of the digitalization process.







%Discuss the consequences of these assumptions in terms of how digital transformation and approaches to managing it are portrayed in the two books.
%-------------------------------------------------------------------------------------------
\section{Consequences of assumptions in digital transformation} \label{sec:Sec2}

\begin{CountingDefinition}[Definitions]{def:defdef}
	
	\begin{tabular}{lp{13.5cm}}
		
		Text
		
	\end{tabular}
	
	
\end{CountingDefinition}






















%Use the results of 1 and 2, and your own example of an organization and context, to describe where, when and why the approach of either Venkatraman or Westerman et al, would not be suitable
%-------------------------------------------------------------------------------------------
\section{Constraints of conventional approaches} \label{sec:Sec3}

\begin{CountingDefinition}[Definitions]{def:defdef}
	
	\begin{tabular}{lp{13.5cm}}
		
		Text
		
	\end{tabular}
	
	
\end{CountingDefinition}




	
	\newpage
	\addcontentsline{toc}{section}{References}
	\begin{spacing}{0.9}
		\printbibliography
	\end{spacing}



\appendix
\newpage
%-------------------------------------------------------------------------------------------
\section{Example companies in \textit{Leading digital}} \label{a:LD}


\begin{figure}[h!]
	\centering
	\includegraphics[width=0.8\textwidth]{images/LD_Table.png}
	\label{fig:LD_table}
	\caption{Companies that were mentioned as examples in \textit{Leading digital} \cite{leadingdigital}}
\end{figure}
	
	
	
	
	
	




\newpage
%-------------------------------------------------------------------------------------------	
\section{Example companies in \textit{Digital matrix}} \label{b:DM}


\begin{figure}[h!]
	\centering
	\includegraphics[width=0.7\textwidth]{images/DM_Table.png}
	\label{fig:DM_table}
	\caption{Companies that were mentioned as examples in \textit{Digital matrix} \cite{digitalmatrix}}
\end{figure}






	
	
	
	
	
	
\end{document}